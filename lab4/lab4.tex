\documentclass[12pt]{article}

\usepackage[utf8x]{inputenc}
\usepackage[T1, T2A]{fontenc}
\usepackage{fullpage}
\usepackage{multicol,multirow}
\usepackage{tabularx}
\usepackage{ulem}
\usepackage{listings} 
\usepackage[english,russian]{babel}
\usepackage{tikz}
\usepackage{pgfplots}
\usepackage{indentfirst}

\parindent=1cm
\makeatletter
\newcommand{\rindex}[2][\imki@jobname]{%
    \index[#1]{\detokenize{#2}}%
}
\makeatother
\newcolumntype{P}[1]{>{\raggedbottom\arraybackslash}p{#1}}

\hfuzz=10000pt
\vbadness10000
\linespread{1}
\lstset{
basicstyle=\ttfamily,
frame=single
}

\begin{document}

\section*{Лабораторная работа №\,4 по курсу Логическое программирование}

Выполнил студент группы М8О-307Б \textit{Лопатин Александр}.

\subsection*{Задание}
Написать программу на Прологе, запросы к которой будут выглядеть следующим
образом:\\
Вариант №3:\\
Реализовать синтаксический анализатор арифметического выражения и
вычислить его числовое значение. В выражении допустимы операции +,-,*,/,
степень \^{}. Учитывать приоритеты операций.

\subsection*{Введение}

2 основных подхода к обработке языков -- статистический и лингвистический. В основе статистического подхода лежит предположение, что содержание текста отражается наиболее часто встречающимися словами. Суть этого анализа заключается в подсчете количества вхождений слов в текст. Лингвистический подход основан на лингвистическом анализе, который представляет собой 4 уровня анализа входных данных: графематический (отдельные слова), морфологический (морфологические характеристики слов), синтаксический (зависимости слов в предложении), семантический (смысл высказывания).

\subsection*{Принцип решения}
Обходим список, проверяя, встретилась ли комбинация "операнд + оператор + операнд". Если встретилась, то считаем значение этого выражения и заменяем его полученным числом. Для операций сложения, вычитания, деления и умножения можно идти слево направо, но для операции возведения в степень нужно проходить справа налево (т.к. выражения x\^{}y\^{}z считается как x\^{}(y\^{}z) ). Для вычисления значения ввсего выражения проходим по списку и
вычисляем сначала все операции возведения в степень, потом операция деления/умножения, 
и уже потом операции сложения и вычитания. Для того, чтобы учесть проход справа налево
для операции возведения степень я для простоты реверсирую список с помощью предиката reverse, который я написал для лабораторной работы №1. Предикат вычисления значения выражения выглядит следующим образом:
\begin{lstlisting}
calculate(L, X):-
    reverse(L, L1, []),
    calc_pows([], L1, L2),
    reverse(L2, L3, []),
    calc_divs([], L3, L4),
    calc_mults([], L4, L5), 
    calc_zero(L5, X).
\end{lstlisting}
Предикаты вычисления для каждого оператора очень похожи друг для друга, поэтому приведу пример только предиката вычисления возведения в степень:
\begin{lstlisting}
calc_pows(L, [X,Y,Z|T], C):-
    pow(Y), R is Z**X, calc_pows(L, [R|T], C).
calc_pows(L, [X, Y, Z|T], C):-
    append(L, [X, Y], L1), calc_pows(L1, [Z|T], C).
calc_pows(L, [X], L1):- 
    append(L, [X], L1).
\end{lstlisting}
\subsection*{Результат работы программы}
Пример вычисления некоторых выражений:
\begin{lstlisting}
?- calculate([5, '+', 3, '^',2], X).
X = 14
?- calculate([1, '-', 3, '^',3, '^',2, +, 12356], X).
X = -7326
?- calculate([1, '+', 2, '*', 4, '/', 3, '-', 8], X).
X = -4.333333333333334
\end{lstlisting}
\subsection*{Вывод}
На прологе гораздо проще писать синтаксические и морфологические разборы естественно-языковых текстов, чем на императивных языках программирования. Пролог предлагает совершенно иные возможности для написания программ, нежели популярные императивные и функциональные языки программирования. Из минусов решения подобных задач искусственного интеллекта можно выделить тот факт, что для их решения требуется некоторая база данных, которую приходится набивать руками.
\end{document}