\documentclass[12pt]{article}

\usepackage[utf8x]{inputenc}
\usepackage[T1, T2A]{fontenc}
\usepackage{fullpage}
\usepackage{multicol,multirow}
\usepackage{tabularx}
\usepackage{ulem}
\usepackage{listings} 
\usepackage[english,russian]{babel}
\usepackage{tikz}
\usepackage{pgfplots}
\usepackage{indentfirst}

\parindent=1cm
\makeatletter
\newcommand{\rindex}[2][\imki@jobname]{%
    \index[#1]{\detokenize{#2}}%
}
\makeatother
\newcolumntype{P}[1]{>{\raggedbottom\arraybackslash}p{#1}}

\hfuzz=10000pt
\vbadness10000
\linespread{1}
\lstset{
basicstyle=\ttfamily,
frame=single
}

\begin{document}

\section*{Лабораторная работа №\,2 по курсу Логическое программирование}

Выполнил студент группы М8О-307Б \textit{Лопатин Александр}.

\subsection*{Задание}
Написать и отладить Пролог-программу решения логической задачи в соответствии с номером варинта.\\
Вариант №3:\\
Как то раз случай свел в купе астронома, поэта , прозаика и драматурга. Это были Алексеев, Борисов, Константинов и Дмитриев. Оказалось, что каждый из них взял с собой книгу написанную одним из пассажиров этого купе. Алексеев и Борисов углубились в чтениепредварительно обменявшись книгами. Поэт читал пьесу, прозаик - очень молоой человек, выпустившись свою книгу, говорил что он никогда и ни чего не читал по астрономии. Борисов купил одно из произведений Дмитриева. Никто из пассажиров не читал свои книги. Что читал каждый из них, кто кем был?

\subsection*{Введение}

Существует 2 основных подхода к решению логических задач: метод порождения и проверок и метод ветвей и границ. Они оба перебирают некоторый набор решений. Суть первого метода состоит в том, что некоторый предикат генерирует множество исходных данных, которые затем проверяются другими предикатами на предмет соответствия условию задачи. В случае неуспеха происходит возврат и рассмотрение следующего решения, в случае успеха полученное решение возвращается пользователю или используется дальше. Второй метод можно противопоставить первому. В методе ветвей и границ значительные части возможных решений отсекаются на раннем этапе выполнения или вообще не генерируются. Например, можно использовать предикат member одновременно для генерации и для проверки, таким образом генерируются не все варианты решений, а какое-то их подмножество. Очевидно, что программа, написанная с помощью второго метода, будет работать быстрее.

\subsection*{Принцип решения}
В ходе решения используется предикат unique - предикат, показывающий, что в списке нет повторяющихся элементов. Также был вынесен отдельно предикат check, который проверяет,
нет ли в списке пассажиров те, кто читает или купил то, что сам написал:
\begin{lstlisting}
check([]):-!.
check([passenger(_, XRead, XBuy, XWrite)|T]):-
  !, not(XRead = XWrite), not(XBuy = XWrite), check(T).
\end{lstlisting}
Задаем начальные условия задачи: каждый написал книгу, купил книгу и читает книгу:
\begin{lstlisting}
Solve = [passenger(alekseev, XRead, XBuy, XWrite),
          passenger(borisov, YRead, YBuy, YWrite),
          passenger(konstantinov, ZRead, ZBuy, ZWrite), 
          passenger(dmitriev, WRead, WBuy, WWrite)],
  writebook(XWrite), writebook(YWrite),
  writebook(ZWrite), writebook(WWrite),
  unique([XWrite, YWrite, ZWrite, WWrite]),
  
  writebook(XBuy), writebook(YBuy),
  writebook(ZBuy), writebook(WBuy),
  unique([XBuy, YBuy, ZBuy, WBuy]),
  
  writebook(XRead), writebook(YRead),
  writebook(ZRead), writebook(WRead),
  unique([XRead, YRead, ZRead, WRead]),
\end{lstlisting}

Далее, используя стандартный предикат
member задаем оставшиеся условия задачи: поэт читает пьесу:
\begin{lstlisting}
member(passenger(_, piece, _, poetry), Solve),
\end{lstlisting}
Прозаик не читает и не покупад астрономию:
\begin{lstlisting}
not(member(passenger(_, astronomy, _, prose), Solve)),
not(member(passenger(_, _, astronomy, prose), Solve)),
\end{lstlisting}
Алексеев и Борисов обменялись книгами:
\begin{lstlisting}
  member(passenger(alekseev, AlekseevRead, AlekseevBuy, _), Solve),
  member(passenger(borisov, AlekseevBuy, AlekseevRead, _), Solve),
\end{lstlisting}
Борисов купил произведение Дмитриева:
\begin{lstlisting}
  member(passenger(dmitriev, _, _, DmitrievWrite), Solve),
  member(passenger(borisov, _, DmitrievWrite, _), Solve).
\end{lstlisting}

\subsection*{Вывод}
В ходе данной лабораторной работы я разобрался, как устроен и как работает язык программирования Prolog, реализовал несложные предикаты для работы со
списками. 
\end{document}