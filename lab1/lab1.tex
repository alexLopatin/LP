\documentclass[12pt]{article}

\usepackage[utf8x]{inputenc}
\usepackage[T1, T2A]{fontenc}
\usepackage{fullpage}
\usepackage{multicol,multirow}
\usepackage{tabularx}
\usepackage{ulem}
\usepackage{listings} 
\usepackage[english,russian]{babel}
\usepackage{tikz}
\usepackage{pgfplots}
\usepackage{indentfirst}

\parindent=1cm
\makeatletter
\newcommand{\rindex}[2][\imki@jobname]{%
    \index[#1]{\detokenize{#2}}%
}
\makeatother
\newcolumntype{P}[1]{>{\raggedbottom\arraybackslash}p{#1}}

\hfuzz=10000pt
\vbadness10000
\linespread{1}
\lstset{
basicstyle=\ttfamily,
frame=single
}

\begin{document}

\section*{Лабораторная работа №\,1 по курсу Логическое программирование}

Выполнил студент группы М8О-307Б \textit{Лопатин Александр}.

\subsection*{Введение}
Связный список — структура данных, состоящая из узлов. Узел содержит данные и ссылку (указатель, связку) на один или два соседних узла. Списки языка Prolog являются односвязными, т.е. каждый узел содержит лишь одну ссылку.\\

В языке Prolog программист не сталкивается с явной работой с указателями в узлах, однако ему нужно иметь общее представление о списках, т.к. являясь основной структурой данных в функциональных и логических языках, они обладают рядом существенных отличий от массивов, используемых в императивных языках (таких как С++, Java, Pascal). В частности, элемент данных может быть очень быстро добавлен или удален из начала односвязного списка. Однако операция произвольного доступа (обращения к n-ному элементу) в списках выполняется гораздо дольше чем в массивах, т.к. требует n операций перехода по ссылкам.\\

При работе с односвязными списками необходимо выделять первый узел (называемый головой списка), остальные узлы (составляющие хвост списка) можно получить передвигаясь по указателям вплоть до последнего узла. Хвост списка является таким же списком, как и исходный, поэтому обрабатывается аналогичным образом (рекурсивно).\\


\subsection*{Реализация стандартных предикатов}
Предикат вычисления длины списка:
\begin{lstlisting}
list_length([], 0).
list_length([_|L], N):-list_length(L, M), N is M + 1.
\end{lstlisting}
Результат работы:
\begin{lstlisting}
?- list_length([1,2,3,4,5,6], X).
X = 6
\end{lstlisting}

Предикат вычисления  конкатeнации списков:
\begin{lstlisting}
append([], L, L).
append([X|L1], L2, [X|L3]):-append(L1, L2, L3).
\end{lstlisting}
Результат работы:
\begin{lstlisting}
?- append([-2, -1], [1,2,3,4,5,6], X).
X = [-2, -1, 1, 2, 3, 4, 5, 6]
\end{lstlisting}

Предикат принадлежности элемента списку:
\begin{lstlisting}
member(X, [X|_]).
member(X, [_|T]):-member(X, T).
\end{lstlisting}
Результат работы:
\begin{lstlisting}
?- member(3, [1,2,3,4,5,6]).
true
?- member(-1, [1,2,3,4,5,6]).
false
\end{lstlisting}

Предикат удаления элемента из списка:
\begin{lstlisting}
remove(X, [X|T], T).
remove(X, [Y|T], [Y|Z]):-remove(X, T, Z).
\end{lstlisting}
Результат работы:
\begin{lstlisting}
?- remove(3, [1,2,3,4,5,6], X).
X = [1, 2, 4, 5, 6]
\end{lstlisting}

Предикат перестановки элементов в списке:
\begin{lstlisting}
permute([], []).
permute(L, [X|T]):-remove(X, L, Y), permute(Y, T).
\end{lstlisting}
Результат работы:
\begin{lstlisting}
?- permute([1,2,3], X).
X = [1, 2, 3]
X = [1, 3, 2]
X = [2, 1, 3]
X = [2, 3, 1]
X = [3, 1, 2]
X = [3, 2, 1]
\end{lstlisting}

Предикат подсписков списка:
\begin{lstlisting}
sublist(S, L):-append(_, L1, L), append(S, _, L1).
\end{lstlisting}
Результат работы:
\begin{lstlisting}
?- sublist(X, [1,2,3]).
X = []
X = [1]
X = [1, 2]
X = [1, 2, 3]
X = []
X = [2]
X = [2, 3]
X = []
X = [3]
X = []
\end{lstlisting}

\subsection*{Предикаты обработки списков}
Вариант 3: реверсирование списка.\\
Без использования стандартных предикатов:
\begin{lstlisting}
reverse([],Z,Z).
 reverse([H|T],Z,Acc) :- reverse(T,Z,[H|Acc]).
\end{lstlisting}
Результат работы:
\begin{lstlisting}
?- reverse([1,2,3], X, []).
X = [3, 2, 1]
\end{lstlisting}
С использованием стандартных предикатов:
\begin{lstlisting}
reverse_std([],Z,Z).
 reverse_std([H|T],Z,Acc) :- 
 append([H], Acc, L1), reverse_std(T, Z, L1).
\end{lstlisting}
Результат работы:
\begin{lstlisting}
?- reverse_std([1,2,3], X, []).
X = [3, 2, 1]
\end{lstlisting}

\subsection*{Предикат обработки числовых списков}
Вариант 3: нахождение максимального элемента в списке.\\
Без использования стандартных предикатов:
\begin{lstlisting}
max([X], X).
max([X|Y], X):-
    max(Y, M), X >= M.
max([X|Y], Z):-
    max(Y, Z), X < Z.
\end{lstlisting}
Результат работы:
\begin{lstlisting}
?- max([1,3,2], X).
X = 3
false
\end{lstlisting}

\subsection*{Пример совместного использования предикатов, реализованных в предыдущих пунктах}
Для примера я выбрал сортировку числового списка - просто на каждом шаге выбираем максимальный элемент, добавляем в новый список, из старого удаляем.\\
\begin{lstlisting}
sort([], L, L).
sort(L1, C, L2):-
    max(L1, M), remove(M, L1, L3), sort(L3, C, [M|L2]).
\end{lstlisting}
Результат работы:
\begin{lstlisting}
?- sort([5,2,1,-5,2,7], X, []).
X = [-5, 1, 2, 2, 5, 7]
\end{lstlisting}

\subsection*{Вывод}
В ходе данной лабораторной работы я разобрался, как устроен и как работает язык программирования Prolog, реализовал несложные предикаты для работы со
списками. 
\end{document}